\documentclass[a3paper,10pt]{extarticle}
% \usepackage[utf8]{inputenc}
\usepackage[mathletters]{ucs}
\usepackage[utf8x]{inputenc}

\usepackage{fancyhdr}

\usepackage[pdftex]{graphicx} % Required for including pictures
\usepackage[pdftex,linkcolor=black,pdfborder={0 0 0}]{hyperref} % Format links for pdf
\usepackage{calc} % To reset the counter in the document after title page
\usepackage{enumitem} % Includes lists

\usepackage{textcomp}
\usepackage{eurosym}

\usepackage{ dsfont } % font za množice
% tabele
\usepackage{array}
\usepackage{wrapfig}
\usepackage{ textcomp }

\usepackage{tikz,forest}
\usetikzlibrary{arrows.meta}

\frenchspacing % No double spacing between sentences
\setlength{\parindent}{0pt}
\setlength{\parskip}{1em}

\usepackage{mathtools}
\usepackage{blkarray, bigstrut} %

\usepackage{amssymb,amsmath,amsthm,amsfonts}
\usepackage{multicol,multirow}
\usepackage{calc}
\usepackage{ifthen}
\usepackage{tabularx}
\usepackage[landscape]{geometry}
\usepackage[formats]{listings}
\usepackage{inconsolata}
%\usepackage[colorlinks=true,citecolor=blue,linkcolor=blue]{hyperref}
%\usepackage{accents}
\usepackage{pdfpages}
\usepackage{lstautogobble}

\newcommand{\vect}[1]{\accentset{\rightharpoonup}{#1}}

\ifthenelse{\lengthtest { \paperwidth = 11in}}
    { \geometry{top=.5in,left=.5in,right=.5in,bottom=.5in} }
    {\ifthenelse{ \lengthtest{ \paperwidth = 297mm}}
        {\geometry{top=1cm,left=1cm,right=1cm,bottom=1cm} }
        {\geometry{top=1cm,left=1cm,right=1cm,bottom=1cm} }
    }
\pagestyle{empty}
\makeatletter
\renewcommand{\section}{\@startsection{section}{1}{0mm}%
                                {-1ex plus -.5ex minus -.2ex}%
                                {0.5ex plus .2ex}%x
                                {\normalfont\large\bfseries}}
\renewcommand{\subsection}{\@startsection{subsection}{2}{0mm}%
                                {-1explus -.5ex minus -.2ex}%
                                {0.5ex plus .2ex}%
                                {\normalfont\normalsize\bfseries}}
\renewcommand{\subsubsection}{\@startsection{subsubsection}{3}{0mm}%
                                {-1ex plus -.5ex minus -.2ex}%
                                {1ex plus .2ex}%
                                {\normalfont\small\bfseries}}
\makeatother
\setcounter{secnumdepth}{0}
%\setlength{\parindent}{0pt}
%\setlength{\parskip}{0pt plus 0.5ex}
\usepackage{textcomp}
\newcommand{\texttildenormal}{\raisebox{0.5ex}{\texttildelow}}

\lstset{basicstyle=\ttfamily, literate={~} {\texttildenormal}{1}}
% -----------------------------------------------------------------------
\begin{document} 

\begin{center}
    \Large{\textbf{SML-NJ}} \\
\end{center}

\begin{multicols}{4}
\setlength{\premulticols}{1pt}
\setlength{\postmulticols}{1pt}
\setlength{\multicolsep}{1pt}
\setlength{\columnsep}{2pt}

\section{Primitivni tipi}
\begin{tabular}{@{}lll@{}}
    \lstinline|()| & \lstinline|unit|\\
    \lstinline|true|, \lstinline|false|& \lstinline|bool|\\
    \lstinline|0|, \lstinline|42|, \lstinline|~123| & \lstinline|int|\\
    \lstinline|3.14| & \lstinline|real|\\
    \lstinline|#"a"| & \lstinline|char|\\
    \lstinline|"Hello World"| & \lstinline|string|\\
    poljuben tip & \lstinline|'a|\\
    primerjalni tip & \lstinline|''a|\\
\end{tabular}

\section{Operatorji}
\begin{tabular}{@{}ll@{}}
    \lstinline|+| & \lstinline|num * num -> num|\\
    \lstinline|-| & \lstinline|num * num -> num|\\
    \lstinline|*| & \lstinline|real * real -> real|\\
    \lstinline|div| & \lstinline|int * int -> int|\\
    \lstinline|/| & \lstinline|real * real -> real|\\
    \lstinline|mod| & \lstinline|int * int -> int|\\
    \lstinline|~| & \lstinline|num -> num|\\
    \lstinline|not| & \lstinline|bool -> bool|\\
    \lstinline|andalso| & \lstinline|bool * bool -> bool|\\
    \lstinline|orelse| & \lstinline|bool * bool -> bool|\\
    \lstinline|<| & \lstinline|''a * ''a -> bool|\\
    \lstinline|>| & \lstinline|''a * ''a -> bool|\\
    \lstinline|<=| & \lstinline|''a * ''a -> bool|\\
    \lstinline|>=| & \lstinline|''a * ''a -> bool|\\
    \lstinline|<>| & \lstinline|''a * ''a -> bool|\\
    \lstinline|=| & \lstinline|''a * ''a -> bool|\\
    \lstinline|:=| & \lstinline|'a ref * 'a -> unit|\\
    \lstinline|!| & \lstinline|'a ref -> 'a|\\
    \lstinline|^| & \lstinline|string * string -> string|\\
    \lstinline|@| & \lstinline|'a list * 'a list -> 'a list|\\
\end{tabular}

\section{Terke}
\begin{tabular}{@{}ll@{}}
    \lstinline|(v1, ..., vn)| & \lstinline|'a1 * ... * 'an |\\
    \lstinline|(#"p", ())| & \lstinline| char * unit |\\
    \lstinline|(123, "abc", false)| & \lstinline| int * string * bool |\\
    \lstinline|{1=123, 2="abc", 3=false}| & \lstinline| int * string * bool |\\
\end{tabular}
\section{Zapisi}
\begin{tabular}{@{}ll@{}}
    \lstinline|{k1=v1, ..., kn=vn}| & \lstinline|{k1: 'a1, ..., kn: 'an}| \\
    \lstinline|{ime="Jan", visina: 150}| & \lstinline|{ime: string, visina: int}| \\
\end{tabular}

Do vrednosti v zapisu dostopamo z \lstinline|#kljuc zapis|.

\section{Seznami}
\begin{tabular}{@{}ll@{}}
    \lstinline|[]| & \lstinline|'a list|\\
    \lstinline|hd::tl| & \lstinline|'a list|\\
    \lstinline|[v1, v2, ..., vn]| & \lstinline|'a list|\\
    \lstinline|[1,2,3]| & \lstinline|int list| \\
    \lstinline|1::2::3::[]| & \lstinline|int list| \\
\end{tabular}

Funkcije za delo s seznami:

\begin{tabular}{@{}ll@{}}
    \lstinline|null| & \lstinline|fn : 'a list -> bool|\\
    \lstinline|length| & \lstinline|fn : 'a list -> int|\\
    \lstinline|op @| & \lstinline|fn : 'a list * 'a list -> 'a list|\\
    \lstinline|hd| & \lstinline|fn : 'a list -> 'a|\\
    \lstinline|tl| & \lstinline|fn : 'a list -> 'a list|\\
    \lstinline|List.last| & \lstinline|fn : 'a list -> 'a|\\
    \lstinline|List.nth| & \lstinline|fn : 'a list * int -> 'a|\\
    \lstinline|List.take| & \lstinline|fn : 'a list * int -> 'a list|\\
    \lstinline|List.drop| & \lstinline|fn : 'a list * int -> 'a list|\\
    \lstinline|rev| & \lstinline|fn : 'a list -> 'a list|\\
    \lstinline|map| & \lstinline|fn : ('a -> 'b) -> 'a list -> 'b list|\\
    \lstinline|List.filter| & \lstinline|fn : ('a -> bool) -> 'a list -> 'a list|\\
    \lstinline|foldl| & \lstinline|fn : ('a*'b -> 'b) -> 'b -> 'a list -> 'b|\\
    \lstinline|foldr| & \lstinline|fn : ('a*'b -> 'b) -> 'b -> 'a list -> 'b|\\
\end{tabular}

Razlika med \lstinline|foldl| in \lstinline|foldr|:
\begin{lstlisting}
fun f (x, acc) = ... ;
foldl f 0 [1, 2, 3] = f(3,f(2,f(1,0)))
foldr f 0 [1, 2, 3] = f(1,f(2,f(3,0)))
\end{lstlisting}

\section{Funkcije}
\begin{tabular}{@{}ll@{}}
    \lstinline|fun ime args = ... | & \lstinline|fn: 'a -> 'b |\\
    \lstinline|fun add (a, b) = a + b| & \lstinline|fn: int * int -> int|\\
    \lstinline|fun add a b = a + b| & \lstinline|fn: int -> int -> int|\\
    \lstinline|fn (a, b) => a+b| & \lstinline|fn: int * int -> int|\\
    \lstinline|fun id x = x| & \lstinline|fn: 'a -> 'a|\\
    \lstinline|fn x => x| & \lstinline|fn: 'a -> 'a|\\
\end{tabular}

Z ključno besedo fun lahko definiramo funkcije po vzorcu:
\begin{lstlisting}
fun f vzorec1 = izraz1
  | f vzorec2 = izraz2
  | ...
  | f vzorecn = izrazn
\end{lstlisting}

\section{Opcije}
\begin{tabular}{@{}ll@{}}
    \lstinline|NONE| & \lstinline| 'a option |\\
    \lstinline|SOME v| & \lstinline| 'a option |\\
    \lstinline|SOME 13| & \lstinline| int option |\\
\end{tabular}

\section{Unije tipov}

\begin{lstlisting}
datatype ime_tipa = CONS1 of 'a1
                  | ...
                  | CONSn of 'an
datatype int_or_bool = INT of int 
                     | BOOL of bool
datatype 'a option = NONE | SOME of 'a
datatype 'a list = [] | :: of 'a * 'a list
datatype ('a, 'b) generic = A of 'a 
                          | B of 'b 
                          | AB of 'a * 'b
                          | NONE
datatype oseba = 
    OSEBA of {ime: string, priimek: string}
\end{lstlisting}

\section{Sinonimi za tipe}
\begin{lstlisting}
type ime_sinonima = tip
type oseba = {ime: string, priimek: string}
type 'a seznam = 'a list
\end{lstlisting}

\section{Ujemanje vzorcev}
\begin{lstlisting}
case izraz of
    vzorec1 => izraz1
  | vzorec2 => izraz2
  | ...
  | vzorecn => izrazn
\end{lstlisting}
Vzorec je lahko poljuben konstruktor ali primitivna vrednost. Vzorce lahko gnezdimo. 
V vzrocu lahko uporabimo tudi spremenljivke, ki jih potem uporabimo v izrazu.

V vzorcu lahko uporabljamo le konstruktorje ali vrednosti istega tipa.
\begin{lstlisting}
case sez of
    [] => 0
  | x::xs => xs @ [x]

case opt of
    NONE => 0
  | SOME x => x
\end{lstlisting}

\section{Vzajemna rekurzija}
Pri definiciji funkciji in tipov lahko uporabimo ključno besedo \lstinline|and|, da definiramo več funkcij ali tipov hkrati.

\begin{lstlisting}
fun liho n = 
    if n = 0 then false else sodo (n-1)
and sodo n = 
    if n = 0 then true else liho (n-1)

datatype a = A of b | Aend
and      b = B of a | Bend
\end{lstlisting}

\section{Lokalno okolje}
\begin{lstlisting}
val x = 10;
let
    val y1 = x + 1 (* y1 = 11 *)
    val x = 1   (* zasencimo globalno x *)
    val y2 = x+1 (* y = 2 *)
    val a = 100
    fun f a = a + y2 (* 'a' vzamemo iz args *)
in
    f y1 (* vrne vrednost 11 + 2 = 13  *)
end
\end{lstlisting}

\section{Mutacije}
\begin{lstlisting}
val x = ref 10; (* x = ref 10 *)
x := 20; (* x = ref 20 *)
!x; (* vrne 20 *)
\end{lstlisting}

\section{Izjeme}
Svoj tip izjeme definiramo z \lstinline|exception|: 
\begin{lstlisting}
exception Izjema of string;
\end{lstlisting}
Izjemo sprožimo z \lstinline|raise|.
\begin{lstlisting}
raise (Izjema "Napaka");
\end{lstlisting}

Izjeme ujamemo z \lstinline|handle|:
\begin{lstlisting}
izraz_ki_prozi_izjemo
handle
    vzorec1 => izraz1
  | vzorec2 => izraz2
  | ...
  | vzorecn => izrazn
\end{lstlisting}

Izjeme so tipa \lstinline|exn|.

\section{Moduli}
\begin{lstlisting}
structure ImeModula = struct
    (* definicije val, fun, datatype, ... *)
end
\end{lstlisting}

Do vrednosti v modulu dostopamo z \lstinline|ImeModula.ime|. Naprimer:
\begin{lstlisting}
structure MojModul = struct
    val x = 10
    fun pozdravi () = "Zivjo" ^ Int.toString x
end;

MojModul.x; (* vrne 10 *)
MojModul.pozdravi(); (* vrne "Zivjo10" *)
\end{lstlisting}

Lahko dodamo podpis modula, ki določa katere vrednosti so vidne izven modula.
\begin{lstlisting}
signature MojModulP = sig
    val pozdravi : unit -> string
end

structure MojModul :> MojModulP = struct
    val x = 10
    fun add (a, b) = a + b
    fun pozdravi () = "Zivjo" ^ Int.toString x
end

MojModul.x; (* napaka *)
MojModul.add(1,2); (* napaka *)
MojModul.pozdravi(); (* vrne "Zivjo10" *)
\end{lstlisting}





\end{multicols}
\end{document}