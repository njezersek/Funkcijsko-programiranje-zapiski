\documentclass[a3paper,9pt]{extarticle}
% \usepackage[utf8]{inputenc}
\usepackage[mathletters]{ucs}
\usepackage[utf8x]{inputenc}

\usepackage{fancyhdr}

\usepackage[pdftex]{graphicx} % Required for including pictures
\usepackage[pdftex,linkcolor=black,pdfborder={0 0 0}]{hyperref} % Format links for pdf
\usepackage{calc} % To reset the counter in the document after title page
\usepackage{enumitem} % Includes lists

\usepackage{textcomp}
\usepackage{eurosym}

\usepackage{ dsfont } % font za množice
% tabele
\usepackage{array}
\usepackage{wrapfig}
\usepackage{ textcomp }

\usepackage{tikz,forest}
\usetikzlibrary{arrows.meta}

\frenchspacing % No double spacing between sentences
\setlength{\parindent}{0pt}
\setlength{\parskip}{0.1em}

\usepackage{mathtools}
\usepackage{blkarray, bigstrut} %

\usepackage{amssymb,amsmath,amsthm,amsfonts}
\usepackage{multicol,multirow}
\usepackage{calc}
\usepackage{ifthen}
\usepackage{tabularx}
\usepackage[landscape]{geometry}
\usepackage[formats]{listings}
\usepackage{inconsolata}
%\usepackage[colorlinks=true,citecolor=blue,linkcolor=blue]{hyperref}
%\usepackage{accents}
\usepackage{pdfpages}
\usepackage{lstautogobble}

\newcommand{\vect}[1]{\accentset{\rightharpoonup}{#1}}

\ifthenelse{\lengthtest { \paperwidth = 11in}}
    { \geometry{top=.5in,left=.5in,right=.5in,bottom=.5in} }
    {\ifthenelse{ \lengthtest{ \paperwidth = 297mm}}
        {\geometry{top=1cm,left=1cm,right=1cm,bottom=1cm} }
        {\geometry{top=1cm,left=1cm,right=1cm,bottom=1cm} }
    }
\pagestyle{empty}
\makeatletter
\renewcommand{\section}{\@startsection{section}{1}{0mm}%
                                {-1ex plus -.5ex minus -.2ex}%
                                {0.5ex plus .2ex}%x
                                {\normalfont\large\bfseries}}
\renewcommand{\subsection}{\@startsection{subsection}{2}{0mm}%
                                {-1explus -.5ex minus -.2ex}%
                                {0.5ex plus .2ex}%
                                {\normalfont\normalsize\bfseries}}
\renewcommand{\subsubsection}{\@startsection{subsubsection}{3}{0mm}%
                                {-1ex plus -.5ex minus -.2ex}%
                                {1ex plus .2ex}%
                                {\normalfont\small\bfseries}}
\makeatother
\setcounter{secnumdepth}{0}
%\setlength{\parindent}{0pt}
%\setlength{\parskip}{0pt plus 0.5ex}
\usepackage{textcomp}
\newcommand{\texttildenormal}{\raisebox{0.5ex}{\texttildelow}}

\lstset{
    basicstyle=\ttfamily, 
    literate={~} {\texttildenormal}{1}, 
    literate={→} {$\rightarrow$}{1},
    mathescape=true
}
% -----------------------------------------------------------------------
\begin{document} 

\begin{multicols}{4}
\setlength{\premulticols}{1pt}
\setlength{\postmulticols}{1pt}
\setlength{\multicolsep}{1pt}
\setlength{\columnsep}{2pt}
\begin{center}
    \huge{\textbf{SML-NJ}} \\
\end{center}

\section{Primitivni tipi}
\begin{tabular}{@{}lll@{}}
    \lstinline|()| & \lstinline|unit|\\
    \lstinline|true|, \lstinline|false|& \lstinline|bool|\\
    \lstinline|0|, \lstinline|42|, \lstinline|~123| & \lstinline|int|\\
    \lstinline|3.14| & \lstinline|real|\\
    \lstinline|#"a"| & \lstinline|char|\\
    \lstinline|"Hello World"| & \lstinline|string|\\
    poljuben tip & \lstinline|'a|\\
    primerjalni tip & \lstinline|''a|\\
\end{tabular}

\section{Operatorji}
\begin{tabular}{@{}ll@{}}
    \lstinline|+| & \lstinline|num * num -> num|\\
    \lstinline|-| & \lstinline|num * num -> num|\\
    \lstinline|*| & \lstinline|real * real -> real|\\
    \lstinline|div| & \lstinline|int * int -> int|\\
    \lstinline|/| & \lstinline|real * real -> real|\\
    \lstinline|mod| & \lstinline|int * int -> int|\\
    \lstinline|~| & \lstinline|num -> num|\\
    \lstinline|not| & \lstinline|bool -> bool|\\
    \lstinline|andalso| & \lstinline|bool * bool -> bool|\\
    \lstinline|orelse| & \lstinline|bool * bool -> bool|\\
    \lstinline|<| & \lstinline|''a * ''a -> bool|\\
    \lstinline|>| & \lstinline|''a * ''a -> bool|\\
    \lstinline|<=| & \lstinline|''a * ''a -> bool|\\
    \lstinline|>=| & \lstinline|''a * ''a -> bool|\\
    \lstinline|<>| & \lstinline|''a * ''a -> bool|\\
    \lstinline|=| & \lstinline|''a * ''a -> bool|\\
    \lstinline|:=| & \lstinline|'a ref * 'a -> unit|\\
    \lstinline|!| & \lstinline|'a ref -> 'a|\\
    \lstinline|^| & \lstinline|string * string -> string|\\
    \lstinline|@| & \lstinline|'a list * 'a list -> 'a list|\\
\end{tabular}

\lstinline|num| predstavlja tip \lstinline|int| ali \lstinline|real|.

\section{Terke}
\begin{tabular}{@{}ll@{}}
    \lstinline|(v1, ..., vn)| & \lstinline|'a1 * ... * 'an |\\
    \lstinline|(#"p", ())| & \lstinline| char * unit |\\
    \lstinline|(123, "abc", false)| & \lstinline| int * string * bool |\\
    \lstinline|{1=123, 2="abc", 3=false}| & \lstinline| int * string * bool |\\
\end{tabular}
\section{Zapisi}
\begin{tabular}{@{}ll@{}}
    \lstinline|{k1=v1, ..., kn=vn}| & \lstinline|{k1: 'a1, ..., kn: 'an}| \\
    \lstinline|{ime="Jan", visina: 150}| & \lstinline|{ime: string, visina: int}| \\
\end{tabular}

Do vrednosti v zapisu dostopamo z \lstinline|#kljuc zapis|.

\section{Seznami}
\begin{tabular}{@{}ll@{}}
    \lstinline|[]| & \lstinline|'a list|\\
    \lstinline|hd::tl| & \lstinline|'a list|\\
    \lstinline|[v1, v2, ..., vn]| & \lstinline|'a list|\\
    \lstinline|[1,2,3]| & \lstinline|int list| \\
    \lstinline|1::2::3::[]| & \lstinline|int list| \\
\end{tabular}

Funkcije za delo s seznami:

\begin{tabular}{@{}ll@{}}
    \lstinline|null| & \lstinline|fn : 'a list -> bool|\\
    \lstinline|length| & \lstinline|fn : 'a list -> int|\\
    \lstinline|op @| & \lstinline|fn : 'a list * 'a list -> 'a list|\\
    \lstinline|hd| & \lstinline|fn : 'a list -> 'a|\\
    \lstinline|tl| & \lstinline|fn : 'a list -> 'a list|\\
    \lstinline|List.last| & \lstinline|fn : 'a list -> 'a|\\
    \lstinline|List.nth| & \lstinline|fn : 'a list * int -> 'a|\\
    \lstinline|List.take| & \lstinline|fn : 'a list * int -> 'a list|\\
    \lstinline|List.drop| & \lstinline|fn : 'a list * int -> 'a list|\\
    \lstinline|rev| & \lstinline|fn : 'a list -> 'a list|\\
    \lstinline|map| & \lstinline|fn : ('a -> 'b) -> 'a list -> 'b list|\\
    \lstinline|List.filter| & \lstinline|fn : ('a -> bool) -> 'a list -> 'a list|\\
    \lstinline|foldl| & \lstinline|fn : ('a*'b -> 'b) -> 'b -> 'a list -> 'b|\\
    \lstinline|foldr| & \lstinline|fn : ('a*'b -> 'b) -> 'b -> 'a list -> 'b|\\
\end{tabular}

Razlika med \lstinline|foldl| in \lstinline|foldr|:
\begin{lstlisting}
fun f (x, acc) = ... ;
foldl f 0 [1, 2, 3] = f(3,f(2,f(1,0)))
foldr f 0 [1, 2, 3] = f(1,f(2,f(3,0)))
\end{lstlisting}

\section{Funkcije}
\begin{tabular}{@{}ll@{}}
    \lstinline|fun ime args = ... | & \lstinline|fn: 'a -> 'b |\\
    \lstinline|fun add (a, b) = a + b| & \lstinline|fn: int * int -> int|\\
    \lstinline|fun add a b = a + b| & \lstinline|fn: int -> int -> int|\\
    \lstinline|fn (a, b) => a+b| & \lstinline|fn: int * int -> int|\\
    \lstinline|fun id x = x| & \lstinline|fn: 'a -> 'a|\\
    \lstinline|fn x => x| & \lstinline|fn: 'a -> 'a|\\
\end{tabular}

Z ključno besedo fun lahko definiramo funkcije po vzorcu:
\begin{lstlisting}
fun f vzorec1 = izraz1
  | f vzorec2 = izraz2
  | ...
  | f vzorecn = izrazn
\end{lstlisting}

\section{Opcije}
\begin{tabular}{@{}ll@{}}
    \lstinline|NONE| & \lstinline| 'a option |\\
    \lstinline|SOME v| & \lstinline| 'a option |\\
    \lstinline|SOME 13| & \lstinline| int option |\\
\end{tabular}

\section{Unije tipov}

\begin{lstlisting}
datatype ime_tipa = CONS1 of 'a1
                  | ...
                  | CONSn of 'an
datatype int_or_bool = INT of int 
                     | BOOL of bool
datatype 'a option = NONE | SOME of 'a
datatype 'a list = [] | :: of 'a * 'a list
datatype ('a, 'b) generic = A of 'a 
                          | B of 'b 
                          | AB of 'a * 'b
                          | NONE
datatype oseba = 
    OSEBA of {ime: string, priimek: string}
\end{lstlisting}

\section{Sinonimi za tipe}
\begin{lstlisting}
type ime_sinonima = tip
type oseba = {ime: string, priimek: string}
type 'a seznam = 'a list
\end{lstlisting}

\section{Ujemanje vzorcev}
\begin{lstlisting}
case izraz of
    vzorec1 => izraz1
  | vzorec2 => izraz2
  | ...
  | vzorecn => izrazn
\end{lstlisting}
Vzorec je lahko poljuben konstruktor ali primitivna vrednost. Vzorce lahko gnezdimo. 
V vzrocu lahko uporabimo tudi spremenljivke, ki jih potem uporabimo v izrazu.

V vzorcu lahko uporabljamo le konstruktorje ali vrednosti istega tipa.
\begin{lstlisting}
case sez of
    [] => 0
  | x::xs => xs @ [x]

case opt of
    NONE => 0
  | SOME x => x
\end{lstlisting}

\section{Vzajemna rekurzija}
Pri definiciji funkciji in tipov lahko uporabimo ključno besedo \lstinline|and|, da definiramo več funkcij ali tipov hkrati.

\begin{lstlisting}
fun liho n = 
    if n = 0 then false else sodo (n-1)
and sodo n = 
    if n = 0 then true else liho (n-1)

datatype a = A of b | Aend
and      b = B of a | Bend
\end{lstlisting}

\section{Lokalno okolje}
\begin{lstlisting}
val x = 10;
let
    val y1 = x + 1 (* y1 = 11 *)
    val x = 1   (* zasencimo globalno x *)
    val y2 = x+1 (* y = 2 *)
    val a = 100
    fun f a = a + y2 (* 'a' vzamemo iz args *)
in
    f y1 (* vrne vrednost 11 + 2 = 13  *)
end
\end{lstlisting}

\section{Mutacije}
\begin{lstlisting}
val x = ref 10; (* x = ref 10 : int ref *)
x := 20; (* x = ref 20 : int ref *)
!x; (* vrne 20 : int *)
\end{lstlisting}

\section{Izjeme}
Svoj tip izjeme definiramo z \lstinline|exception|: 
\begin{lstlisting}
exception Izjema of string;
\end{lstlisting}
Izjemo sprožimo z \lstinline|raise|.
\begin{lstlisting}
raise (Izjema "Napaka");
\end{lstlisting}

Izjeme ujamemo z \lstinline|handle|:
\begin{lstlisting}
izraz_ki_prozi_izjemo
handle
    vzorec1 => izraz1
  | vzorec2 => izraz2
  | ...
  | vzorecn => izrazn
\end{lstlisting}

Izjeme so tipa \lstinline|exn|.

\section{Moduli}
\begin{lstlisting}
structure ImeModula = struct
    (* definicije val, fun, datatype, ... *)
end
\end{lstlisting}

Do vrednosti v modulu dostopamo z \lstinline|ImeModula.ime|. Naprimer:
\begin{lstlisting}
structure MojModul = struct
    val x = 10
    fun pozdravi () = "Zivjo" ^ Int.toString x
end;

MojModul.x; (* vrne 10 *)
MojModul.pozdravi(); (* vrne "Zivjo10" *)
\end{lstlisting}

Lahko dodamo podpis modula, ki določa katere vrednosti so vidne izven modula.
\begin{lstlisting}
signature MojModulP = sig
    val pozdravi : unit -> string
end

structure MojModul :> MojModulP = struct
    val x = 10
    fun add (a, b) = a + b
    fun pozdravi () = "Zivjo" ^ Int.toString x
end

MojModul.x; (* napaka *)
MojModul.add(1,2); (* napaka *)
MojModul.pozdravi(); (* vrne "Zivjo10" *)
\end{lstlisting}

\section{Prednosti leksikalnega dosega}
\begin{itemize}
    \item Imena spremenljivk v funkciji so neodvisna od imen zunanjih spremenljivk
    \item Funkcija je neodvisna od imen uporabljenih spremenljivk
    \item Tip funkcije lahko določimo ob njeni deklaraciji
    \item Ovojnica shrani podatke, ki jih potrebuje za kasnejšo izvedbo.
\end{itemize}
\columnbreak
\begin{center}
    \huge{\textbf{PYTHON}} \\
\end{center}
\section{Anonimna funkcija}
\begin{lstlisting}[language=Python]
f = lambda x: x + 1
f = lambda x, y: x + y
\end{lstlisting}
\section{Dekoratorji}
\begin{lstlisting}[language=Python]
def decor(f):
    def wrapper(*args, **kwargs):
        print(args, kwargs)
        f(*args, **kwargs)
        print('end')
    return wrapper

@decor
def say_hi(name, times=1):
    print(f"hi {name}! "*times)

say_hi('mom', times=3)

# ('mom',) {'times': 3}
# hi mom! hi mom! hi mom! 
# end
\end{lstlisting}

\section{Iteratorji}

Z razredom:
\begin{lstlisting}[language=Python]
class MojIterator:
    def __init__(self, args):
        # inicializacija
    
    def __iter__(self):
        return self

    def __next__(self):
        # vrne naslednji element 
          ali pa raise StopIteration

for x in MojIterator(args):
    pass
\end{lstlisting}

Z generatorjem:
\begin{lstlisting}[language=Python]
def moj_generator(args):
    # inicializacija
    while True:
        # yield naslednji element
        # ali pa return za konec

for x in moj_generator(args):
    pass        
\end{lstlisting}

Z sestavljanjem iteratorjev:
\begin{lstlisting}[language=Python]
moj_iterator = (x**2 for x in range(10))

for x in moj_iterator:
    pass
\end{lstlisting}

\section{Povezava med funkcijskim in objektnim prog.}
\begin{tabular}{@{}p{4.3cm}p{4.3cm}@{}}
    \textbf{Funkcijsko} & \textbf{Objektno}\\ \hline
    Če dodamo novo funkcijo, moramo v njej pokriti vse konstruktorje za podatkovni tip. &
    Če dodamo novo funkcijo, jo moramo implementirati za vse podrazrede. \\ \hline
    Če dodamo nov konstruktor (podtip), ga moramo v vseh funkcijah pokriti. &
    Če dodamo nov (pod)razred moramo v njem implementirati vse funkcije. \\ \hline
    \emph{Združujemo po funkcijah} & \emph{Združujemo po razredih (tipih)} \\
\end{tabular}

\end{multicols}


\pagebreak
\begin{multicols}{4}
    \setlength{\premulticols}{1pt}
    \setlength{\postmulticols}{1pt}
    \setlength{\multicolsep}{1pt}
    \setlength{\columnsep}{2pt}

\begin{center}
    \huge{\textbf{RACKET}} \\
\end{center}

\section{Definicije}
Na začetku vsake datoteke moramo dodati: \lstinline|#lang racket|

Definicija vrednosti:
\begin{lstlisting}
(define ime vrednost)
\end{lstlisting}

\section{Funkcije}
Definicija funkcije:
\begin{lstlisting}
(define (ime arg1 arg2 ... argn) izraz)
\end{lstlisting}

Definicija funkcije z poljubnim številom argumentov:
\begin{lstlisting}
(define (ime . args) izraz)

(ime 1 2 3) ; args = '(1 2 3)
\end{lstlisting}

Definicija funkcije z poimenovanimi argumenti in privzetimi vrednostmi. Nepoimenovani argumenti s privzetimi vrednostmi morajo biti za ostalimi nepoimenovanimi argumenti.
\begin{lstlisting}
(define (f #:ime_a a b #:ime_c c) izraz )
(f #ime_c 3 2 #ime_a 1) ; a = 1, b = 2, c = 3

(define (f #:ime_a [a 1] [b 2]) izraz )
(f) ; a = 1, b = 2
\end{lstlisting}

Anonimna funkcija:
\begin{lstlisting}
(lambda (arg1 arg2 ... argn) izraz)
(lambda args izraz)
\end{lstlisting}

\section{Uporabne funkcije}
Aritmetika

\begin{tabular}{@{}ll@{}}
    \lstinline|(+ a1 ... an)| & $a_1 + \dots + a_n$  \\
    \lstinline|(- a1 ... an)| & $a_1 - \dots - a_n$  \\
    \lstinline|(* a1 ... an)| & $a_1 * \dots * a_n$  \\
    \lstinline|(/ a1 ... an)| & $a_1 * \dots * a_n$  \\
    \lstinline|(remainder a b)| & ostanek pri deljenju $a/b$  \\
    \lstinline|(modulo a b)| & $a \mod b$  \\
    \lstinline|(abs x)| & $|x|$  \\
    \lstinline|(max a1 ... an)| & $\max(a_1, \dots, a_n)$  \\
    \lstinline|(min a1 ... an)| & $\min(a_1, \dots, a_n)$  \\
    \lstinline|(gcd a b)| & največji skupni delitelj  \\
    \lstinline|(lcm a b)| & najmanjši skupni večkratnik  \\
    \lstinline|(round x)| & zaokroži  \\
    \lstinline|(floor x)| & zaokroži navzdol  \\
    \lstinline|(ceiling x)| & zaokroži navzgor  \\
    \lstinline|(truncate x)| & odreži decimalni del  \\
    \lstinline|(sqrt x)| & $\sqrt{x}$  \\
    \lstinline|(expt x y)| & $x^y$  \\
    \lstinline|sin| \dots \lstinline|atan| & trigonometrične funkcije  \\
\end{tabular}\\

Primerjava:

\begin{tabular}{@{}ll@{}}
    \lstinline|(eq? a1 ... an)| & primerjava po referenci  \\
    \lstinline|(= a1 ... an)| & $a_1 = \dots = a_n$  \\
    \lstinline|(> a1 ... an)| & $a_1 > \dots > a_n$  \\
    \lstinline|(>= a1 ... an)| & $a_1 \geq \dots \geq a_n$  \\
    \lstinline|(< a1 ... an)| & $a_1 < \dots < a_n$  \\
    \lstinline|(<= a1 ... an)| & $a_1 \leq \dots \leq a_n$  \\
\end{tabular}\\

Logične operacije:

\begin{tabular}{@{}ll@{}}
    \lstinline|(not a)| & $\neg a$  \\
    \lstinline|(and a1 ... an)| & $a_1 \land \dots \land a_n$  \\
    \lstinline|(or a1 ... an)| & $a_1 \lor \dots \lor a_n$  \\
    \lstinline|(xor a1 ... an)| & $a_1 \oplus \dots \oplus a_n$  \\
\end{tabular}\\

Nizi:

\begin{tabular}{@{}ll@{}}
    \lstinline|(string? a)| & ali je \lstinline|a| niz  \\
    \lstinline|(string-length a)| & dolžina niza  \\
    \lstinline|(string-ref a i)| & $i$-ti znak niza  \\
    \lstinline|(string-append a1 ... an)| & združi nize  \\
\end{tabular}\\


\section{Nadzor toka}
\begin{lstlisting}
(if pogoj potem sicer)
\end{lstlisting}
Stavek \lstinline|cond| vrne vrednost prvega izraza, pri katerem je resničen pogoj (ne vrača \lstinline|#f|).
\begin{lstlisting}
(cond [pogoj1 izraz1]
      [pogoj2 izraz2]
      ...
      [else izrazn])
\end{lstlisting}
\section*{Ujemanje vzorcev}
\begin{lstlisting}
(match izraz
    (vzorec1 izraz1)
    (vzorec2 izraz2)
    ...
    (else izrazn))
\end{lstlisting}

\section{Pari in seznami}

\begin{tabular}{@{}p{3.2cm}l@{}}
    Vrednost \lstinline|'()| & \lstinline|null| : \lstinline|null?|\\
    Ustvari par & \lstinline|(cons a d)| $\rightarrow$ \lstinline|pair?|\\
    Vrni prvi el. & \lstinline|(car p)| $\rightarrow$ \lstinline|any/c|\\
    Vrni drugi el. & \lstinline|(cdr p)| $\rightarrow$ \lstinline|any/c|\\
    Ali je \lstinline|v| par & \lstinline|(pair? v)| $\rightarrow$ \lstinline|boolean?|\\
    Ali je \lstinline|v| enak \lstinline|'()| & \lstinline|(null? v)| $\rightarrow$ \lstinline|boolean?|\\
    Ustvari seznam & \lstinline|(list v ...)| $\rightarrow$ \lstinline|list?|\\
    Ali je \lstinline|v| seznam & \lstinline|(list? v)| $\rightarrow$ \lstinline|boolean?|\\
    Dolžina seznama & \lstinline|(length lst)| $\rightarrow$ \lstinline|integer?|\\
    Obrni seznam & \lstinline|(reverse lst)| $\rightarrow$ \lstinline|list?|\\
    Vrni \lstinline|i|-ti el. seznama & \lstinline|(list-ref lst i)| $\rightarrow$ \lstinline|any/c|\\
    Uporabi \lstinline|proc| na vsakem el. v \lstinline|lst| & \lstinline|(map proc lst ...+)| $\rightarrow$ \lstinline|list?|\\
    Zloži iz leve & \lstinline|(foldl proc init lst ...+)| $\rightarrow$ \lstinline|list?|\\
    Zloži iz desne & \lstinline|(foldr proc init lst ...+)| $\rightarrow$ \lstinline|list?|\\
    Vrni seznam brez elementov za katere \lstinline|pred| vrne \lstinline|#f| & \lstinline|(filter pred lst)| $\rightarrow$ \lstinline|list?|\\
\end{tabular}

Seznam je sestavljen iz parov, kjer je drugi element seznam ali \lstinline|'()|.
\begin{lstlisting}
(list 1 2 3) $\equiv$ (cons 1 (cons 2 (cons 3 '())))
\end{lstlisting}

Primer uporabe \lstinline|foldl| in \lstinline|foldr|:
\begin{lstlisting}
(define (f x acc) ...)
(foldl f init (list 1 2 3)) 
    ; f(3, f(2, f(1, init)))
(foldr f init (list 1 2 3)) 
    ; f(1, f(2, f(3, init)))
\end{lstlisting}

Funkcije \lstinline|map|, \lstinline|foldl| in \lstinline|foldr| lahko uporabljamo na več seznamih hkrati.
\begin{lstlisting}
(map + (list 1 2 3) (list 4 5 6)) 
    ; vrne (list 5 7 9)
(foldl 
    (lambda (x y acc) (+ acc (max x y))) 
    0 (list 10 2 3) (list 4 20 5)) ; vrne 35 
\end{lstlisting}

\section{Lokalno okolje}
\begin{tabular}{@{}lp{8cm}@{}}
    \lstinline|let| & izrazi se evalvirajo v oklju \textit{pred} izrazom \lstinline|let|\\
    \lstinline|let*| & izrazi se evalvirajo po vrsti in okolje se posodablja (kot SML)\\
    \lstinline|letrec| & izrazi se evalvirajo v okolju, ki že vesebuje vse deklaracije naštete v \lstinline|letrec| (kot \lstinline|and| v SML)\\
    \lstinline|define| & deluje enako kot \lstinline|letrec|, le drugačna sintaksa
\end{tabular}

Primer uporabe \lstinline|let|:
\begin{lstlisting}
(define x 100)
(let ([x 10] [y x]) (cons x y)) 
    ; vrne (10 . 100)
\end{lstlisting}

Primer uporabe \lstinline|let*|:
\begin{lstlisting}
(define x 100)
(let* ([x 10] [y x]) (cons x y)) 
    ; vrne (10 . 10)
\end{lstlisting}

Primer uporabe \lstinline|letrec|:
\begin{lstlisting}
(letrec 
    ([x (lambda () y)] [y 123]) 
    (cons (x) y)) 
    ; vrne (123 . 123)
\end{lstlisting}

\section{Zakasnjena evalvacija}
Zakasnjena evalvacija:
\begin{lstlisting}
(thunk izraz) $\equiv$ (lambda () izraz)
\end{lstlisting}

Zakasnitev in sprožitev z uporabo \lstinline|delay| in \lstinline|force|:
\begin{lstlisting}
(define x 
    (delay (begin 
        (print "dolg izracun") 
        "rezultat")))
(force x) 
    ; izpise "dolg izracun" in vrne "rezultat"
(force x) 
    ; vrne (shranjen) "rezultat"
\end{lstlisting}

Funkcija \lstinline|delay| ustvari objekt tipa \lstinline|promise|, ki hrani izračunano vrednost.

\section{Tokovi}
Tok je funkcija, ki vrača par, v katerem je privi element vrednost drugi pa tok.

Primer:
\begin{lstlisting}
(define enke (cons 1 (lambda () enke)))
(define naravna 
    (letrec (
        [f (lambda (x) 
            (cons x (thunk (f (+ 1 x)))))]) 
        (f 1)))
\end{lstlisting}

Funkcija, ki izpiše prvih \lstinline|n| elementov toka:
\begin{lstlisting}
(define (izpisi n tok)
    (if (> n 1) 
        (begin
          (displayln (car tok))
          (izpisi (- n 1) ((cdr tok))))
        (displayln (car tok))))
\end{lstlisting}

Funkcija, ki izpsuje elemente toka dokler velja \lstinline|pogoj|:
\begin{lstlisting}
(define (izppog tok pogoj)
    (cond [(pogoj (car tok)) (begin
            (displayln (car tok))
            (izppog ((cdr tok)) pogoj))]
          [#t #t]))
\end{lstlisting}

\section{Memoizacija}
Rezultate funkcije shranimo v tabelo in jih uporabimo, če je funkcija ponovno klicana z istimi argumenti.

Za tabelo lahko uporabimo \lstinline|hash| ali seznam parov (kjuč, vrednost).

\begin{lstlisting}
(define f (letrec 
    ([memo (make-hash)])
    (lambda (x) 
        (if (hash-has-key? memo x) 
            (hash-ref memo x) 
            (let ([rezultat ...])
                (hash-set! memo x rezultat)
                rezultat
            )))))
\end{lstlisting}

\begin{tabular}{@{}p{3.3cm}l@{}}
    Ustvari hash tabelo & \lstinline|(make-hash)| $\rightarrow$ \lstinline|hash?|\\
    Ali je \lstinline|v| hash tabela & \lstinline|(hash? v)| $\rightarrow$ \lstinline|boolean?|\\
    Vrni vrednost za ključ \lstinline|k| & \lstinline|(hash-ref h k)| $\rightarrow$ \lstinline|any/c|\\
    Nastavi vrednost za ključ \lstinline|k| na \lstinline|v| & \lstinline|(hash-set! h k v)| $\rightarrow$ \lstinline|void?|\\
    Ali ima \lstinline|h| ključ \lstinline|k| & \lstinline|(hash-has-key? h k)| $\rightarrow$ \lstinline|boolean?|\\
    Izbriši ključ \lstinline|k| iz \lstinline|h| & \lstinline|(hash-remove! h k)| $\rightarrow$ \lstinline|void?|\\
\end{tabular}

\section{Makri}
\begin{lstlisting}
(define-syntax ime-makra
    (syntax-rules (kljucna-beseda1 ...) 
        [(ime-makra vzorec) izraz]
        ...
    )
)
\end{lstlisting}

\lstinline|vzorec| lahko vsebuje spremenljivke, ki se vežejo na vrednosti v izrazu in ključne besede, ki so naštete v \lstinline|syntax-rules|.

Primer makra:
\begin{lstlisting}
(define-syntax mojif
    (syntax-rules (then else)
      [(mojif p then et) 
        (if p et '())]
      [(mojif p then et else ef)
        (if p et ef)]
      [(mojif p1 then e1t elif p2 e2t else ef)
        (if p1 e1t if p2 e2t ef)]
    ))

(mojif #t then 123 else 456) ; vrne 123
\end{lstlisting}

\section{Lastni podatkovni tipi}
\begin{lstlisting}
(struct mojtip 
    (polje1 polje2 ... poljen) 
    #:transparent ; polja so vidna v REPL
    #:mutable ; polja lahko spreminjamo
)
\end{lstlisting}
Avtomatsko dobimo funkcije:

\begin{tabular}{@{}ll@{}}
    Konstruktor & \lstinline|(mojtip v1 v2 ... vn)| $\rightarrow$ \lstinline|mojtip|\\
    Preverjanje tipa & \lstinline|(mojtip? v)| $\rightarrow$ \lstinline|boolean?|\\
    Vrednost polja & \lstinline|(mojtip-polje v)| $\rightarrow$ \lstinline|any/c|\\
    Nastavi polje* & \lstinline|(set-mojtip-polje! v vrednost)| $\rightarrow$ \lstinline|void?|\\
\end{tabular}

\emph{* če je \lstinline|#:mutable|.}

\section{Preverjanje}

\begin{tabular}{@{}p{2.5cm}p{6.5cm}@{}}
\textbf{statično} & 
    preverjanje pred izvajanjem programa
    (ne preveja semantičnih napak, izjem, pravilnosti artimetičnih operatij (deljenje z 0), ...)\\
\textbf{dinamično} & preverjanje med izvajanjem programa\\
\end{tabular}
\section{Trdnost in polnost sistema tipov}
\begin{tabular}{@{}p{2.5cm}p{6.5cm}@{}}
    \textbf{pozitiven} & program z napako \\
    \textbf{negativen} & program brez napake\\
\end{tabular}\\

\begin{tabular}{@{}p{2.5cm}p{6.5cm}@{}}
    \textbf{trden} (sound) & sistem nikoli ne sprejme pozitivnega programa (ima pa lažno pozitivne primere) \\
    \textbf{poln} (comoplete) & sistem nikoli ne zavrne negativnega programa (ima pa lažno negativne primere) \\
\end{tabular}

\section{Šibki in močni sistemi tipov}
Ne obstaja neka uveljavljena definicija. Racket ima \emph{šibki} sistem tipov. SML ima \emph{močen} sistem tipov.


\end{multicols}
\end{document}